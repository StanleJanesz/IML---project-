\documentclass[11pt, letterpaper]{article}
\usepackage{graphicx}
\usepackage[export]{adjustbox}
\usepackage{hyperref}
\usepackage{xcolor}
\graphicspath{{images/}}
\usepackage[a4paper, total={6in, 8in}]{geometry}
\usepackage{tabularx}
\usepackage{polski}
\usepackage{float}


\title{\textbf {Tytuł} \\ \large Podtytuł}

\author{Łukasz Górski \\ Piotr Iśtok \\ Piotr Jacak \\ Stanisław Janowicz \\ Marcin Falkowski}

\date{Styczeń 2025}


\begin{document}
\maketitle

\newpage
\tableofcontents
\newpage

\section{Wprowadzenie i ogólna koncepcja}
\section{Data cleaning and pre-processing}
\subsection{Preprocessing}
\subsection{Cleaning}
In machine learning applications that involve audio data, such as speaker identification, the quality of the input signal is critical. Background noise can obscure important voice features, which can lead to reduced model performance. One way of overcoming this issue and improving SNR (signal to noise ratio) is denoising data using spectral gating. It works by computing a spectrogram of a signal (and optionally a noise signal) and estimating a noise threshold (or gate) for each frequency band of that signal/noise.
That threshold is used to compute a mask, which gates noise below the frequency-varying threshold. The main reasons why it is a good option for reducing noise are:
\begin{itemize}
    \item \textbf{Preservation of signal features:}
        Spectral gating works in the frequency domain, selectively weakening noise-dominated frequencies, keeping the structure of our target signal (voice) mostly intact.
    \item \textbf{Flexibility:}
        The method uses a noise profile derived from a sample with only background noise, which allows our program to adapt to different environments such as street noise, electrical noise and reverberation. Because of it our code is simple without any noticeable performance drop.
    \item \textbf{Ease of use:}
        This method of denoising is implemented in the free-to-use Python library called "noisereduce", keeping our solution easily reproducible and can be seamlessly integrated into our pipeline.
\end{itemize}
Unfortunately, spectral gating also has its drawbacks. The biggest one is that, for it to work there needs to be a noise sample provided, which can be for us sometimes impossible to distinguish. That's why in that case our denoising step in pipeline will use band-stop filter, implemented with Butterworth filter. The main advantages of it are:
\begin{itemize}
    \item \textbf{No need for a noisy sample:}
        This filter always removes lower frequencies, so urban sounds can be easily eliminated, without the need of noisy sample.
    \item \textbf{Preservation of higher frequencies:}
        The Butterworth filter's characteristic smooth frequency response ensures minimal distortion in the preserved higher frequencies.
    \item \textbf{Preprocessing Simplicity:}
        This approach is computationally efficient and easy to implement.
\end{itemize}
The reason why it is not used all the time is that it has many limitations. It is not ideal for broadband noise. It may not effectively remove all of it. The other one is that it overlaps with voice frequencies, so it can also attenuate it resulting in reduced intelligibility.\\We also considered using deep learning for audio cleaning, as it has shown great potential by leveraging neural networks to learn complex mappings between noisy and clean audio signals. However, we came to the conclusion that it is too advanced a tool to remove such simple noises and two methods described before are efficient enough.
\subsection{Spectrograms}
In speaker recognition tasks, the selection of an appropriate feature extraction method is crucial. The acoustic properties of human speech vary widely and capturing these nuances effectively is essential for accurate classification. Among many methods for converting raw audio signals into a form suitable for machine learning, it was imposed on us to use spectrograms. We created three types, however in the final version only one of three listed down below methods is used.
\begin{itemize}
    \item \textbf{Regular Spectrogram}
        This method computes the spectrogram by applying the Short-Time Fourier Transform (Stft) to the audio signal. It visualizes how the signal's frequency content varies over time. While this approach provides an accurate representation of raw frequency components, it does not consider the perceptual differences in how humans interpret sound.
    \item \textbf{Spectrogram with adjustable parameters and dB scale}
        This method allowed us to changed parameters such as FTT size, window size and hop size. However tweaking those parameters didn't bring any improvements. The biggest advantage of this method compare to the previous one is dB scaling of spectrogram. Because of that more features were clearly shown.
    \item \textbf{Mel Spectrogram}
        The mel spectrogram method transforms the audio signal into a representation that closely matches the human ear's sensitivity to frequency. It maps the frequencies to the mel scale, which emphasizes lower frequencies (important for speech) while de-emphasizing higher, less perceptible frequencies. By focusing on perceptually relevant features, the mel spectrogram strikes a balance between simplicity, efficiency, and effectiveness. It directly produces features well-suited for downstream models without requiring extensive manual adjustments. To no surprise, this method we had the best results so it is being used in final version.
\end{itemize}
Additionally all spectrograms are stored in grayscale to take up as little space as possible without loosing important data for further usage.
\end{document}